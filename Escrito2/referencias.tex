\renewcommand{\refname}{Referencias}
\begin{thebibliography}{99}

\bibitem{cppn} STANLEY, K.O.(2007). ``Compositional pattern producing networks: A novel abstraction of development''. Genetic Programming and Evolvable Machines Special Issue on Developmental Systems, 8(2):131?16.

\bibitem{neat} STANLEY, K.O., and MIIKKULAINEN, R. (2002). ``Evolving neural networks through augmenting topologies''. Evolutionary Computation, 10(2):99-127.

\bibitem{hyperneat} STANLEY, K.O., D'AMBROSIO, D., and GAUSI, J. (2009). ``A hypercube-based encoding for evolving large-scale neural networks''. Artificial Life, 15(2):185-212.

\bibitem{zykov} ZYKOV, V., BONGARD, J., and LIPSON, H. (2004). ``Evolving Dynamic Gaits on a Physical Robot'', Proceedings of Genetic and Evolutionary Computation Conference, Late Breaking Paper, GECCO.

\bibitem{hornby} HORNBY, G.S., TAKAMURA, S., YAMAMOTO, T., and FUJITA M. (2005). ``Autonomous Evolution of Dynamic Gaits with Two Quadruped Robots''.

\bibitem{clune} CLUNE, J., BECKMAN, B.E., OFRIA, C., and PENNOCK, R.T. (2009). ``Evolving coordinated quadruped gaits with the HyperNEAT generative encoding'', In Proceedings of the IEEE Congress on Evolutionary Computing.

\bibitem{yosinski} YOSINSKI, J., CLUNE, J., HIDALGO, D., NGUYEN, S., ZAGAL, J.C., and LIPSON, H. (2011). ``Evolving robot gaits in hardware: the HyperNEAT generative encoding vs. parameter optimization'', In Proceedings of the 20th European Conference on Artificial Life.

\bibitem{lee} LEE, S., YOSINSKI, J., GLETTE, K., and CLUNE, J. (2013). ``Evolving Gaits for Physical Robots with the HyperNEAT Generative Encoding: The Benefits of Simulation''.

\bibitem{caamano} CAAMA�O, P., BELLAS, F., and DURO, R. (2014). ``\(\tau\)-NEAT: Initial experiments in precise temporal processing through neuroevolution'', International Joint Conference on Neural Networks. 

\bibitem{vrep} Virtual Robot Experimentation Plataform, Coppelia Robotics, Switzerland.\\ $\<$http://www.coppeliarobotics.com/$\>$

\bibitem{deep} Deep learning, From Wikipedia, the free enciclopedia.\\
$\<$https://en.wikipedia.org/wiki/Deep\_learning$\>$

\bibitem{rl} RICHARD S SUTTON and ANDREW G BARTO. Reinforcement learning: An introduction, volume 1. Cambridge Univ Press, 1998.

\bibitem{churchuland} CHURCHLAND, P.M. (1986). Some reductive strategies in cognitive neurobiology. Mind, 95:279-309.

\bibitem{dyn} Dynamixel Motors, Robotis, Korea.\\ $\<$http://www.robotis.com/xe/dynamixel\_en$\>$

\end{thebibliography}