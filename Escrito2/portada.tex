\pagestyle{empty}

\begin{center}

\large \textbf{UNIVERSIDAD T\'ECNICA FEDERICO SANTA MAR\'IA}

\vspace{3mm}

\normalsize DEPARTAMENTO DE ELECTR\'ONICA

\vspace{40mm}

\Large {\bf ESTIMACI\'ON DE ESTADO EN SISTEMAS OBSERVADOS A TRAV\'ES DE CANALES CON P\'ERDIDA DE DATOS.}

\vspace{32mm}

\normalsize
Tesis de Grado presentada por

\vspace{2mm}

\large \textbf{Miguel Andr\'es Sol\'is Cid.}

\vspace{10mm}

\normalsize
como requisito parcial para optar al t\'itulo de

\vspace{2mm}

\textbf{Ingeniero Civil Electr\'onico}

\vspace{2mm}

y al grado de

\vspace{2mm}

\textbf{Mag\'ister en Ciencias de la Ingenier\'ia Electr\'onica}

\vspace{5mm}

Profesor Gu\'ia

%\vspace{2mm}

Dr. Eduardo Silva Vera

%\vspace{5mm}

%Profesor Co-guía

%\vspace{2mm}

%Dr. Mario Salgado Brocal

\vspace{10mm}

Valpara\'iso, 2012.


\end{center}

\cleardoublepage

\vspace{5mm}

\noindent T\'ITULO DE LA TESIS:

\vspace{5mm}

\noindent{\large {\bf ESTIMACI\'ON DE ESTADO EN SISTEMAS OBSERVADOS A TRAV\'ES DE CANALES CON P\'ERDIDA DE DATOS}}
\vspace{20mm}

\noindent AUTOR:

\vspace{5mm}

\noindent{\large {\bf Miguel Andr\'es Sol\'is Cid.}}

\vspace{15mm}

\noindent TRABAJO DE TESIS, presentado en cumplimiento parcial de
los requisitos para el t\'itulo de Ingeniero Civil Electr\'onico y el
grado de Mag\'ister en Ingenier\'ia Electr\'onica de la Universidad
T\'ecnica Federico Santa Mar\'ia.

\vspace{15mm}

\begin{tabular}{p{60mm}c}
Dr. Eduardo Silva V. & \rule{60mm}{1pt} \\
& \\
& \\
& \\
Dr. Mario Salgado & \rule{60mm}{1pt} \\
& \\
& \\
& \\
Dr. Alejandro Rojas & \rule{60mm}{1pt} \\
&
\end{tabular}

\vspace{5mm}

\hfill Valpara\'iso, Agosto de 2012.

%\cleardoublepage

\cleardoublepage
\newpage
\thispagestyle{empty}
\mbox{}
\cleardoublepage

\vspace{50mm}

\begin{flushright}
%  {\emph{Aqui van dedicatorias}}

% \vspace{5mm}
%  {\emph{Dedicado a}
%  }

\end{flushright}
