\pagestyle{empty}

\begin{center}

\large \textbf{UNIVERSIDAD T\'ECNICA FEDERICO SANTA MAR\'IA}

\vspace{3mm}

\normalsize DEPARTAMENTO DE ELECTR\'ONICA

\vspace{40mm}

\Large {\bf ``\(\tau\)-HYPERNEAT: RETARDOS DE TIEMPO EN UNA RED HYPERNEAT PARA APRENDIZAJE DE CAMINATAS EN ROBOTS CON EXTREMIDADES M\'OVILES''\\}

\vspace{32mm}

\normalsize
Tesis de Grado presentada por

\vspace{2mm}

\large \textbf{Oscar Andr\'e Silva Mu\~noz}

\vspace{10mm}

\normalsize
como requisito parcial para optar al t\'itulo de

\vspace{2mm}

\textbf{Ingeniero Civil Electr\'onico}

\vspace{5mm}

Profesor Gu\'ia

%\vspace{2mm}

Dra. Maria Jos\'e Escobar Silva

\vspace{5mm}

Profesor Correferente

\vspace{2mm}

Dr. Fernando Auat Cheein

\vspace{10mm}

Valpara\'iso, 2016.


\end{center}

\cleardoublepage

\vspace{5mm}

\noindent T\'ITULO DE LA TESIS:

\vspace{5mm}

\noindent{\large {\bf ``\(\tau\)-HYPERNEAT: RETARDOS DE TIEMPO EN UNA RED HYPERNEAT PARA APRENDIZAJE DE CAMINATAS EN ROBOTS CON EXTREMIDADES M\'OVILES''\\}}
\vspace{20mm}

\noindent AUTOR:

\vspace{5mm}

\noindent{\large {\bf Oscar Andr\'e Silva Mu\~noz.}}

\vspace{15mm}

\noindent TRABAJO DE TESIS, presentado en cumplimiento parcial de
los requisitos para el t\'itulo de Ingeniero Civil Electr\'onico de la Universidad
T\'ecnica Federico Santa Mar\'ia.

\vspace{15mm}

\begin{tabular}{p{60mm}c}
Dra. Maria Jos\'e Escobar & \rule{60mm}{1pt} \\
& \\
& \\
& \\
Dr. Fernando Auat & \rule{60mm}{1pt} \\
&
\end{tabular}

\vspace{5mm}

\hfill Valpara\'iso, Septiembre de 2016.

%\cleardoublepage

\cleardoublepage
\newpage
\thispagestyle{empty}
\mbox{}
\cleardoublepage

\vspace{50mm}

\begin{flushright}
%  {\emph{Aqui van dedicatorias}}

% \vspace{5mm}
%  {\emph{Dedicado a}
%  }

\end{flushright}
