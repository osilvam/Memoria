El presente trabajo de tesis se enmarca dentro del dise\~no y an\'alisis de sistemas de estimaci\'on sobre redes de comunicaci\'on. En tal escenario, uno se encuentra con problemas de dise\~no donde los objetivos de estimaci\'on deben balancearse con restricciones impuestas por el canal de comunicaci\'on que se est\'e utilizando. 
Estas restricciones pueden manifestarse como p\'erdida de datos, como limitaciones en la tasa de transferencia de datos, o tambi\'en como retardos en el env\'io de la se\~nal medida.

Este trabajo se encuentra motivado por el reciente inter\'es en redes que incorporan sensores inal\'ambricos, los que miden variables locales para ser enviadas a nodos centrales que realizan su procesamiento. Dado que los sensores involucrados usualmente tienen una potencia disponible limitada, transmitir con alta potencia no es una opci\'on viable. Por lo tanto, las restricciones de comunicaci\'on juegan un papel importante en este tipo de sistemas.

Existen distintos tipos de estimadores, dependiendo de si se tiene acceso o no al estado del canal. Otra diferencia importante al momento de distinguir los distintos tipos de estimadores, se refiere a si el filtro pertenece a la clase de Sistemas con Saltos Markovianos (MJLS - Markov Jump Linear Systems) o no. Luego de hechas estas diferencias, cabe notar si el filtro es \'optimo o sub\'optimo dependiendo de si la varianza del error de estimaci\'on que se obtiene, es la m\'inima posible. Ya realizadas estas distinciones, otro aspecto relevante se refiere a si el filtro converge a uno estacionario o no.

En la presente Tesis se estudia el problema de estimaci\'on en base a pérdida de las observaciones, generando observaciones intermitentes. Se dise\~na un filtro estacionario que minimiza la varianza del error de estimaci\'on, proveyendo, adem\'as, de una estimaci\'on \'optima de los datos perdidos. Lo anterior distingue la propuesta de los esquemas de compensaci\'on utilizados com\'unmente, que consisten en reemplazar el dato perdido por la \'ultima muestra recibida de forma exitosa o, simplemente, reemplazar dichas muestras por ceros.

Para derivar nuestros resultados, se explota un resultado reciente en donde se provee de una equivalencia entre un sistema que env\'ia datos a trav\'es de un canal que pierde datos, y un sistema lineal e invariante en el tiempo con ruido aditivo, sujeto a una restricci\'on de se\~nal a ruido (SNR). En cuanto a la estructura de la planta a considerar, el trabajo se ha realizado sobre el supuesto que la planta es estable y puede tener m\'ultiples entradas y m\'ultiples salidas, proveyendo expresiones cerradas para el caso particular en que la planta cuenta con una salida escalar. 

A lo largo de este documento se presentan diferentes ejemplos que permiten ilustrar los desarrollos realizados, y comparar \'estos frente a otros propuestos en la literatura.

\section*{Palabras Clave:} Filtraje \'optimo, Estimaci\'on de estado, Filtro estacionario, MJLS, P\'erdida de datos, SNR, Ruido aditivo, Varianza m\'inima, Sistemas de tiempo discreto.