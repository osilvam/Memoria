La generaci\'on de caminatas en robots con extremidades m\'oviles es una importante tarea para permitir el correcto desplazamiento de robots en distintos escenarios, ya que el dise\~no manual de estas involucra un gran desaf\'io debido al elevado uso de tiempo y procesamiento. En este proyecto de memoria se ha investigado sobre un popular m\'etodo de neuroevoluci\'on utilizado para esta tarea, llamado HyperNEAT, Hipercubo basado en el Aumento de Topolog\'ias. HyperNEAT es un generador de codificaciones que evoluciona redes neuronales artificiales haciendo uso de los principios del algoritmo de Neuroevoluci\'on basado en el Aumento de Topolog\'ias (NEAT). Es una novedosa t\'ecnica para la evoluci\'on de redes neuronales de gran escala usando las regularidades geom\'etricas del problema descritas por la red. En base a este m\'etodo es que esta memoria propone la implementaci\'on del m\'etodo llamado \(\tau\)-HyperNEAT, incorporando conceptos temporales en una red neuronal HyperNEAT, incluyendo retardos de tiempo adicionales a los pesos en las conexiones entre neuronas,y permitiendo as\'i la generaci\'on de caminatas con comportamientos y caracter\'isticas m\'as cercanas a caminatas vistas en la naturaleza. Posterior a la implementaci\'on de ambos m\'etodos, estos son puestos a prueba en la generaci\'on de caminatas de dos robots con distinto n\'umero de grados de libertad. El an\'alisis comparativo de los resultados revela que, con respecto a las variables cuantitativas del experimento, no existe una diferencia relevante en el desempe\~no obtenido entre HyperNEAT y \(\tau\)-HyperNEAT, pero no as\'i desde el punto de vista cualitativo, obteni\'endose notorias diferencias en la coordinaci\'on de los movimiento de las extremidades de cada robot, produciendo caminatas m\'as naturales y complejas.

\section*{Palabras Clave:} Red neuronal artificial, Neuroevoluci\'on, HyperNEAT, \(\tau\)-HyperNEAT, Aprendizaje de caminatas.