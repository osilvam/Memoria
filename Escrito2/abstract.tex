Gait generation for legged robots is an important task to allow an appropiate displacement in different scenarios, since hand-tuning design involves a big challenge due to the high computational efforts translated into processing power and time. This project study a popular neuroevolution method called HyperNEAT, which stands for Hypercube-based Neuroevolution of Augmenting Topologies. HyperNEAT is a generative encoding mechanism which evolves artificial neural networks using the principles of the algorithm of Neuroevolution of Augmenting Topologies (NEAT). It is a new technique for evolve large scale artificial neural networks using geometric regularities of the problem present in the network. Based on this method, this report proposes the implementation of the method called \(\tau\)-HyperNEAT, incorporating temporal concepts in a HyperNEAT neural network, introducing additional time delays in neural connections between neurons, allowing gait generation with behaviours and characteristics more similar to gaits found in nature. Subsequent to the implementation of both methods, these are tested in gaits generation of two robots with different number of degrees of freedom. The comparative analysis of the results reveals that, regarding quantitative variables of the experiment, there is no significant difference in performance obtained between HyperNEAT and \(\tau\)-HyperNEAT, but not from the qualitative point of view, obtaining notable differences in coordinating the movement of the limbs of each robot, producing more natural and complex gaits.

\section*{Keywords:} Artificial neural network, neuroevolution, HyperNEAT, \(\tau\)-HyperNEAT, Learning gaits.