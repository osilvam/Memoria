\section{Equivalencia entre sistemas LTI y conmutados}\label{ap:demo-teo1}

A continuaci\'on se demuestra una versi\'on generalizada del Teorema \ref{teo:equivalencia}. El desarrollo se basa en \cite{sebatesis}.

Considere un sistema LTI $N$ descrito por
\begin{align}
x(k+1) & = A x(k) + B_d d(k) + B_u u(k), \quad k\in \N_0, \quad x(0)=x_0, \\
  e(k) & = C_e x(k) + D_{de} d(k) + D_{ue} u(k), \\
  v(k) & = C_v x(k) + D_{dv} d(k),
\end{align}
donde $x_0$ corresponde a una variable aleatoria de segundo orden con varianza $P_0\geq 0$, $d$ corresponde a ruido blanco de media cero con varianza $P_d>0$ y no correlacionado con el estado inicial $x_0$, $u$ corresponde a una entrada auxiliar, mientras que $e$ y $v$ son salidas del sistema. Suponga, adem\'as, que $D_{dv}D_{dv}^{\intercal}>0$.

Supondremos ahora que existe un canal de comunicaci\'on entre $v$ y $u$. Primero, suponemos que la relaci\'on entre $v$ y $u$ est\'a dada por
\begin{align}\label{eq:erasure-apendix}
u(k) = \theta(k) v(k),
\end{align}
donde $\theta$ se encuentra definido en la Secci\'on \ref{sec:def_prob_mimo}, y es independiente de $(x_o,d)$ (ver Figura \ref{fig:general}(a)). Es claro que la varianza del estado de $N$ en la Figura \ref{fig:general}(a) satisface (ver demostraci\'on de Lema 6.3 en \cite{elia05}):
\begin{align}\label{eq:Px}
P_x(k+1) = A_pP_x(k)A_p^{\intercal} + B_pP_dB_p^{\intercal} + p(1-p)B_uP_v(k) B_u^{\intercal} \quad k\in\N_0, \quad P_x(0)=P_o,
\end{align}
donde $A_p\treq A+pB_uC_v$, $B_p\treq B_d + pB_uD_{dv}$ y la varianza $P_v(k)$ de $v$ satisface
\begin{align}\label{eq:Pv}
P_v(k) = C_v P_x(k) C_v^{\intercal} + D_{dv} P_d D_{dv}^{\intercal}.
\end{align}
Adem\'as, se tiene que
\begin{align}\label{eq:Pe}
P_e(k) = C_p P_x(k) C_p^{\intercal} + D_p P_d D_p^{\intercal} + p(1-p) D_{ue} P_v(k) D_{ue}^{\intercal},
\end{align}
con $C_p\treq C_e+pD_{ue}C_v$ y $D_p\treq D_{de} +pD_{ue}D_{dv}$.

Suponga ahora que el canal de comunicaci\'on entre $v$ y $u$ est\'a descrito por
\begin{align}\label{eq:aditivo-apendix}
u(k) = n(k) + p\, v(k)
\end{align}
\newpage
\noindent
donde $n$ est\'a definido en la Secci\'on \ref{sec:diseno_mimo}, y no est\'a correlacionado con $(x_o,d)$ (ver Figura \ref{fig:general}(b)).
\begin{figure}
\centering
\scalebox{1}{\input {./figuras/general.pstex_t}}
\caption{Sistema auxiliar $N$ con retroalimentaci\'on a trav\'es de (a) un canal de borrado, y (b) un canal con ruido aditivo y ganancia $p$.}
\label{fig:general}
\end{figure}
Para evitar confusi\'on, y ser consistente con la notaci\'on usad en el Capitulo \ref{cap:estimacion_mimo}, se agregara el sub-\'indice $L$ a todas las se\~nales cuyas estad\'isticas se vean afectadas por el cambio en el enlace entre $v$ y $u$.

Es f\'acil ver que la varianza del estado de $N$ en la Figura \ref{fig:general}(b), as\'i como la varianza de las salidas $v$ y $e$, satisfacen (se us\'o \eqref{eq:aditivo-apendix}):
\begin{align}
\label{eq:PxL} P_{x_L}(k+1) &= A_pP_{x_L}(k)A_p^{\intercal} + B_pP_dB_p^{\intercal} + B_u P_n B_u^{\intercal} \quad k\in\N_0, \quad P_{x_L}(0)=P_o \\
\label{eq:PvL} P_{v_L}(k) & = C_v P_{x_L}(k) C_v^{\intercal} + D_{dv}P_dD_{dv}^{\intercal} \\
\label{eq:PeL} P_{e_L}(k) & = C_p P_{x_L}(k) C_p^{\intercal} + D_p P_d D_p^{\intercal} + D_{ue} P_n D_{ue}^{\intercal}
\end{align}
donde $A_p, B_p, C_p$ y $D_p$ fueron definidas anteriormente.

\begin{lema}\label{lema:estabilidad-apendice}{\ \\}
Considere los sistemas de la Figura \ref{fig:general}(a) y (b) bajo las suposiciones realizadas anteriormente. Entonces, el sistema con saltos de la Figura \ref{fig:general}(a) es MSS si y s\'olo si el sistema LTI de la Figura \ref{fig:general}(b) es estable y existe una elecci\'on (positiva semidefinida) para $P_n$ tal que se cumple la restricci\'on $P_n=p(1-p)P_{v_L}$, donde $P_{v_L}$ es la varianza estacionaria de $v_L$.
\end{lema}

\begin{proof}{\ \\}
\begin{enumerate}
\item Utilizando el Corolario 3.26 de \cite{cofrma05} y \eqref{eq:Px}--\eqref{eq:Pv}, se tiene que el sistema con saltos de la Figura \ref{fig:general}(a) es MSS si y s\'olo si la expresi\'on
\begin{align}\label{eq:lyapunov-generalizada}
P_x = A_p P_x A_p^{\intercal} + p(1-p)B_u C_v P_x C_v^{\intercal} B_u^{\intercal} + B_p P_d B_p^{\intercal} + p(1-p) B_u D_{dv} P_d D_{dv}^{\intercal} B_u^{\intercal}
\end{align}
admite una soluci\'on \'unica y positiva semidefinida $P_x$, que corresponde a la varianza estacionaria del estado. Esta condici\'on es equivalente (usando el Corolario 3.26 de \cite{cofrma05}) a la existencia de $M>0$ tal que
\begin{align}\label{eq:condicion-mss}
M-A_p M A_p^{\intercal} - p(1-p)B_uC_v M C_v^{\intercal}B_u^{\intercal} >0.
\end{align}
Concluimos que, si el sistema con saltos es MSS, entonces existe un $M>0$ tal que se cumple que $M-A_pMA_p^{\intercal}>0$ y el sistema LTI de la Figura \ref{fig:general}(b) ser\'a estable. (Note de \eqref{eq:PxL} que la matriz ``A'' del sistema LTI de la Figura \ref{fig:general}(b) est\'a dada precisamente por $A_p$.)

Suponga que el sistema con saltos es MSS y elija
\begin{align}\label{eq:eleccion}
P_n = P_n^o \treq p(1-p) \left( C_v P_{x} C_v^{\intercal} + D_{dv} P_d D_{dv}^{\intercal} \right)
\end{align}
donde $P_x$ satisface \eqref{eq:lyapunov-generalizada}. Dado que, adem\'as, el sistema LTI de la Figura \ref{fig:general}(b) es estable, entonces la varianza estacionaria $P_{x_L}$ de $x_L$ corresponde a la \'unica soluci\'on positiva semidefinida de
\begin{align}
\label{eq:lyap-lti-0} P_{x_L} &= A_p P_{x_L} A_p^{\intercal} + B_p P_d B_p^{\intercal} + B_u P_n^o B_u^{\intercal} \\
\nonumber   &= A_p P_{x_L} A_p^{\intercal} + p(1-p)B_u C_v P_x C_v^{\intercal} B_u^{\intercal} + B_p P_d B_p^{\intercal} + \\ \label{eq:lyap-lti} & \hspace{60mm} p(1-p) B_u D_{dv} P_d D_{dv}^{\intercal} B_u^{\intercal}
\end{align}
Dado que las soluciones de \eqref{eq:lyapunov-generalizada} y \eqref{eq:lyap-lti} son \'unicas, se tiene que $P_{x_L}=P_x$. Por lo tanto, se ha comprobado que existe una elecci\'on de $P_n$ para la cual se cumple $P_n = p(1-p) \big( C_v P_{x_L} C_v^{\intercal}$ + $D_{dv} P_d
D_{dv}^{\intercal} \big) = p(1-p) P_{v_L}$ (ver \eqref{eq:eleccion} y \eqref{eq:PvL}).
%....................................................................................
\item Si el sistema LTI de la Figura \ref{fig:general}(b) es estable y existe un $P_n$ tal que se cumple $P_n=p(1-p)P_{v_L}$, entonces
\begin{align}
P_n = p(1-p) \left( C_v P_{x_L} C_v^{\intercal} + D_{dv} P_d
D_{dv}^{\intercal} \right)
\end{align}
donde $P_{x_L}$ satisface \eqref{eq:lyap-lti-0} con $P_n^o= P_n$, y existe un $P>0$ tal que
\begin{align}\label{eq:lyap-estabilidad}
P-A_pPA_p^{\intercal}>0.
\end{align}
Note que nuestras suposiciones garantizan que se cumple $D_{dv} P_d D_{dv}^{\intercal}>0$, lo que implica que $P_n > p(1-p) C_v P_{x_L}C_v^{\intercal}$ y, por lo tanto, que existe un $\epsilon>0$ tal que
\begin{align}\label{eq:Pq-epsilon}
P_n > p(1-p) C_v \left( P_{x_L} + \epsilon P \right) C_v^{\intercal}
\end{align}
donde $P$ satisface \eqref{eq:lyap-estabilidad}.  

Luego, usando \eqref{eq:Pq-epsilon}, \eqref{eq:lyap-estabilidad} y \eqref{eq:lyap-lti-0} (con $P_n^o$ reemplazado por $P_n$) se procede como en \cite{linlem04} para mostrar que \eqref{eq:condicion-mss} se cumple con $M = P_{x_L} + \epsilon P > 0$. De hecho,
\begin{align}
& A_p \left( P_{x_L} + \epsilon P \right)A_p^{\intercal} + p(1-p) B_u C_v \left( P_{x_L} + \epsilon
P \right) C_v^{\intercal} B_u^{\intercal} \non \\
 & \qquad \qquad < A_p \left( P_{x_L} + \epsilon P \right) A_p^{\intercal} + B_uP_n B_u^{\intercal} \non \\
 & \qquad \qquad = P_{x_L} - B_pP_dB_p^{\intercal} + \epsilon A_pP A_p^{\intercal} \non \\
 & \qquad \qquad \leq P_{x_L} + \epsilon A_pPA_p^{\intercal} \non \\ \label{eq:demo-ap}
 & \qquad \qquad < P_{x_L} + \epsilon P
\end{align}
donde se debe notar que en la primera desigualdad estricta se ha hecho uso de \eqref{eq:Pq-epsilon}, en la igualdad que sigue se ha usado \eqref{eq:lyap-lti-0}, la desigualdad no estricta es trivial, y la \'ultima desigualdad se obtiene utilizando \eqref{eq:lyap-estabilidad}.
\\ \noindent
Concluimos de \eqref{eq:demo-ap} que el sistema con saltos de la Figura \ref{fig:general}(a) es MSS, completando la demostraci\'on\QED
\end{enumerate}

\end{proof}

\begin{coro}\label{coro:varianzas-apendice}{\ \\}
Considere el sistema y las suposiciones involucradas en el Lema \ref{lema:estabilidad-apendice}.  Si el sistema con saltos de la Figura \ref{fig:general}(a) es MSS o, equivalentemente, si el sistema LTI de la Figura \ref{fig:general}(b) es estable y existe una elecci\'on para $P_n$ tal que $P_n=p(1-p)P_{v_L}$, entonces, para dicha elecci\'on para $P_n$, $P_{\alpha} = P_{\alpha_L}$ para $\alpha\in\{x,e,v\}$.
\end{coro}

\begin{proof}{\ \\}
En la demostraci\'on del Lema \ref{lema:estabilidad-apendice} se mostr\'o que bajo las suposiciones y definiciones all\'i consideradas, se cumple que $P_{x_L}=P_x$.  Luego, la demostraci\'on del Corolario \ref{coro:varianzas-apendice} es inmediata utilizando este hecho, con \eqref{eq:Pe},\eqref{eq:Pv}, \eqref{eq:PeL}, y \eqref{eq:PvL}.
\QED
\end{proof}

El Teorema \ref{teo:equivalencia} corresponde a un caso particular del Lema \ref{lema:estabilidad-apendice} y del Corolario \ref{coro:varianzas-apendice}. 
Cabe notar que, en el Teorema \ref{teo:equivalencia}, el hecho de que $P_{\omega}=I$ y $D_yD_y^{\intercal}>0$ implica que las suposiciones $P_d>0$ y $D_{dv}D_{dv}^{\intercal}>0$ en el Lema \ref{lema:estabilidad-apendice} y el Corolario \ref{coro:varianzas-apendice} se cumplen.

\section{Programaci\'on Fraccional}\label{sec:desp-inf}
A continuaci\'on se muestra un resultado que permite separar el numerador y el denominador de cierto funcional en base a cierto par\'ametro. Este resultado puede ser revisado en mayor detalle en \cite{dinkelbach1967nonlinear}.
\begin{lema}\label{lema:dinkelbach}{\ \\}
Sea $g(x)>0$ una funci\'on real. Considere el funcional $J = \max_{x}{\frac{f(x)}{g(x)}}$, donde $J^*$ corresponde a su valor \'optimo, entonces la expresi\'on
\begin{align*}
J^*&=\frac{f(x^*)}{g(x^*)}\\
&=\max_{x}\{\frac{f(x)}{g(x)}\},
\end{align*}
se cumple si y s\'olo si
\begin{equation}
\max_{x}\{f(x)-J^*g(x)\}=0.
\end{equation}
\end{lema}
Cabe notar que, tal como se enuncia en \cite{dinkelbach1967nonlinear}, este resultado se cumple también si se trabaja con $\min$ en lugar de $\max$.

\section{Expresiones expl\'icitas en estimaci\'on \'optima}\label{sec:snyder}
A continuaci\'on se provee de un resultado que permite caracterizar la varianza del error de estimaci\'on en filtraje lineal \'optimo para procesos estacionarios, cuando se encuentran contaminados con ruido aditivo. 

Sean dos secuencias WSS $\{ m_t\}$ y $\{ n_t\}$ el mensaje y el ruido respectivamente, secuencias que se suponen no correlacionadas entre s\'i, entonces el MSE, es decir, la diferencia cuadr\'atica entre la secuencia que se desea reproducir y la mejor que se puede obtener, cuando se utiliza un filtro $h(e^{j\omega})$ viene dado por
\begin{equation}\label{eq:Esnyder}
E = \min_{h}{\frac{1}{2\pi}\int_{-\pi}^{\pi}{\left(|h(e^{j\omega})-e^{j\omega}|^2f(e^{j\omega})+|h(e^{j\omega})|^2g(e^{j\omega})\right)d\omega}}
\end{equation}
As\'i, el siguiente resultado permite obtener una expresi\'on en forma cerrada para la caracterizaci\'on de $E$.
\begin{lema}\label{lema:snyder}{\ \\}
Sea $g\neq 0$ la PSD (constante) del ruido de cierto sistema, $f$ la PSD de la salida del sistema (la cual corresponde a la entrada del filtro \'optimo en cuesti\'on), y $E$ el error cuadr\'atico medio de estimaci\'on descrito por \eqref{eq:Esnyder}, entonces $E$ se puede escribir como:
\begin{equation}
E=g\left( exp \left( \frac{1}{2\pi}\int_{-\pi}^{\pi}{log\left(1+\frac{f(e^{j\omega})}{g}\right)d\omega}\right)-1\right)
\end{equation}

\end{lema}
Este resultado junto con su correspondiente demostraci\'on se encuentra en mayor detalle en \cite{snyder72-ams}.

\section{Ecuaci\'on Algebraica de Riccati Modificada}\label{sec:mare}
En esta secci\'on se presentar\'an las propiedades de la MARE que son utilizadas con mayor frecuencia dentro de este trabajo de tesis. Mayor detalle sobre estas y otras propiedades de la MARE se pueden encontrar en \cite{schenato2007optimal}.
\begin{lema}
Sea el par\'ametro $\lambda \in (0,1)$, y defina los siguientes operadores
\begin{multline}
L_{\lambda}(K,P) = \lambda A\left(I-KC\right)P\left(I-KC\right)^{\intercal}A^{\intercal}+(1-\lambda)APA^{\intercal}+\lambda AKDD^{\intercal}K^{\intercal}A^{\intercal}\\+BB^{\intercal},
\end{multline}
\begin{equation}
\Phi_{\lambda}(P) = APA^{\intercal}-\lambda APC^{\intercal}\left(CPC^{\intercal}+R\right)^{-1}CPA^{\intercal}+BB^{\intercal},
\end{equation}
donde $A,B,C,D$ como es usual son matrices conocidas y vienen dadas por la representaci\'on en variables de estado de cierto sistema, y $K$ corresponde a la ganancia constante de cierto filtro.

Entonces, se cumplen las siguientes propiedades
\begin{itemize}
\item Si $P_1\geq P_2$, entonces $\Phi_{\lambda}(P_1)\geq \Phi_{\lambda}(P_2)$.
\item Si $\lambda_1 \geq \lambda_2$, entonces $\Phi_{\lambda_1}(P)\leq \Phi_{\lambda_2}(P)$.
\item Si existe $P^*$ tal que $P^* = L_{\lambda}(K,P^*)$, entonces $P^*$ es \'unica y adem\'as es semidefinida positiva.
\item Si $\lambda_1\geq \lambda_2$ y existe $P_1^*,P_2^*$ tal que $P_1^*=\Phi_{\lambda_1}(P_1^*)$ y $P_2^*=\Phi_{\lambda_2}(P_2^*)$, entonces $P_1^*\leq P_2^*$.
\item Si $A$ es estrictamente estable (i.e., todos sus autovalores son estrictamente menor que 1), entonces $P^*=\Phi_{\lambda}(P^*)$ siempre tiene una soluci\'on.
\item Si existe $P^*>0$ y $K$ tal que $P^*=L_{\lambda}(K,P^*)$, entonces $A_c=A\left(I-\lambda KC\right)$ es estrictamente estable.
\end{itemize}
\end{lema}

\newpage
\section{Teorema de los Residuos}
Sea la funci\'on $f(q)$ con $q$ el argumento de la transformada Z, y $q_0$ un punto singular aislado de la funci\'on $f$. En este caso, la funci\'on puede ser representada mediante una serie de Laurent \cite{chubro90}:
\begin{equation}
f(q)=a_n(q-q_0)^n+\ldots+a_2(q-q_0)^2+a_1(q-q_0)+a_0+\frac{b_1}{q-q_0}+\ldots+\frac{b_n}{(q-q_0)^n}
\end{equation}
Se puede probar que
\begin{equation}
b_n=\frac{1}{2\pi j}\oint_{\mathfrak{C}} \frac{f(q)}{(q-q_0)^{-n+1}}dq,
\end{equation}
donde $\mathfrak{C}$ es una curva cerrada orientada en sentido antihorario, que encierra a la singularidad en $q=q_0$. Cuando se tiene $n=1$, puede escribirse
\begin{equation}
b_1=\frac{1}{2\pi j}\oint_{\mathfrak{C}} f(q)dq.
\end{equation}
El numero $b_1$, que es el coeficiente de $1/(q-q_0)$ se llama el residuo de $f$ en el punto singular $q_0$.

\begin{teo}\label{teo:residuos}{\ \\}
Considere las definiciones realizadas anteriormente, entonces el c\'alculo de los residuos de $f(q)$
\begin{equation}
b_1 = Res_{q=q_0}f(q)
\end{equation}
De esta forma, se tiene la relaci\'on
\begin{equation}
\frac{1}{2\pi j}\oint_{\mathfrak{C}} f(q)dq=\sum_{n=0}^kRes_{q=q_n}f(q),
\end{equation}
donde $q_n$ corresponde a las singularidades de $f$ dentro de la curva $\mathfrak{C}$, recorrida en sentido antihorario.
\end{teo}
Un desarrollo m\'as detallado respecto al Teorema de los Residuos se puede encontrar en \cite{chubro90}.