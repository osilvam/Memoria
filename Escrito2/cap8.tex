\section{DISCUSION Y COMPARACION DE LOS RESULTADOS OBTENIDOS}

En este trabajo de memoria se ha implementado un nuevo m�todo de neuroevoluci�n, llamado \(\tau\)-HyperNEAT, con el objetivo de ponerlo a prueba en la tarde de generaci�n de caminatas en robots con extremidades m�viles. En particular, se ha planteado la hip�tesis de que este nuevo m�todo de neuroevoluci�n, al incorporar retardos de tiempo en las conexiones a lo largo de la red principal, tendr� mejoras en su desempe�o final y generar� caminatas con caracter�sticas cualitativas superiores.

Luego de analizar todos los resultados obtenidos a partir de los experimentos realizados es posible concluir lo siguiente.

\begin{enumerate}
\item En la tarea de generaci�n de caminatas, no hubo diferencias cuantitativas en los resultados obtenidos a trav�s de los m�todos HyperNEAT y \(\tau\)-HyperNEAT, obteni�ndose en ambos casos el mismo desempe�o de a cuerdo a las variables involucradas en el proceso de calificaci�n de las caminatas.
\item En la tarea de generaci�n de caminatas, \(\tau\)-HyperNEAT logr� desarrollar, casi en un 100\%, caminatas con movimientos arm�nicos y coordinados, similares a comportamientos vistos en la naturaleza. Esto pudo ser observado en los gr�ficos de las se�ales de los motores generadas por \(\tau\)-HyperNEAT, evidenciandose las diferencias y similitudes de fases vistas entre distintas extremidades y motores de una misma extremidad respectivamente.
\end{enumerate}

Ya que las capacidades de HyperNEAT para generar caminatas con movimientos arm�nicos y coordinados fueron pr�cticamente nulas frente a las capacidades mostradas por el m�todo \(\tau\)-HyperNEAT, es posible afirmar que este nuevo m�todo, al incorporar retardos de tiempo a lo largo de su red principal, logra afrontar y resolver de mejor manera problemas din�micos que involucran variables temporales, como lo es en particular, la generaci�n de caminatas en robots con extremidades m�viles.

\section{TRABAJO FUTURO}

Dados los resultados obtenidos en este trabajo de memoria, se considera que los siguientes
son temas para trabajo futuro, enfocados tanto en la generaci�n de caminatas como en la de cualquier tipo de experimento en donde se puedan aprovechar las capacidades de las redes HyperNEAT y \(\tau\)-HyperNEAT.

\begin{itemize}
\item Verificar las posibles variaciones en el desempe�o del m�todo \(\tau\)-HyperNEAT para distintos valores del retardo m�ximo ($\tau_{max}$) en las conexiones de la red.
\item Experimentar con diferentes configuraciones geom�tricas en el substrato de las redes HyperNEAT y \(\tau\)-HyperNEAT, con el objetivo de explotar al m�ximo las capacidades de ambas redes.
\item Formular y probar nuevas funciones de desempe�o para lograr clasificar de manera �ptima la correcta ejecuci�n de una caminata, favoreciendo siempre el proceso de evoluci�n de caminatas realmente efectivas y as� obtener mejores resultados.
\item Traspasar los resultados obtenidos en simulaciones a un entorno real, emulando de manera correcta las din�micas presentes en cada robot.
\end{itemize}