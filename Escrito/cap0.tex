\section*{Agradecimientos}

Gracias a todos los involucrados en este trabajo, ya sea de forma directa o indirecta.

Gracias a mi familia y amigos.

\newpage

\section*{Resumen}

La generaci�n de caminatas en robots con extremidades m�viles es una importante tarea para permitir el correcto desplazamiento de robots en distintos escenarios, ya que el dise�o manual de estas involucra un gran desaf�o debido al elevado uso de tiempo y procesamiento. En este proyecto de memoria se ha investigado sobre un popular m�todo de neuroevoluci�n utilizado para esta tarea, llamado HyperNEAT, Hipercubo basado en el Aumento de Topolog�as. HyperNEAT es un generador de codificaciones que evoluciona redes neuronales artificiales haciendo uso de los principios del algoritmo de Neuroevoluci�n basado en el Aumento de Topolog�as (NEAT). Es una novedosa t�cnica para la evoluci�n de redes neuronales de gran escala usando las regularidades geom�tricas del problema descritas por la red. En base a este m�todo es que esta memoria propone la implementaci�n del m�todo llamado \(\tau\)-HyperNEAT, incorporando conceptos temporales en una red neuronal HyperNEAT, incluyendo retardos de tiempo adicionales a los pesos en las conexiones entre neuronas,y permitiendo as� la generaci�n de caminatas con comportamientos y caracter�sticas m�s cercanas a caminatas vistas en la naturaleza. Posterior a la implementaci�n de ambos m�todos, estos son puestos a prueba en la generaci�n de caminatas de dos robots con distinto n�mero de grados de libertad. El an�lisis comparativo de los resultados revela que, con respecto a las variables cuantitativas del experimento, no existe una diferencia relevante en el desempe�o obtenido entre HyperNEAT y \(\tau\)-HyperNEAT, pero no as� desde el punto de vista cualitativo, obteni�ndose notorias diferencias en la coordinaci�n de los movimiento de las extremidades de cada robot, produciendo caminatas m�s naturales y complejas.

\section*{Palabras Clave:} Red neuronal artificial, Neuroevoluci\'on, HyperNEAT, \(\tau\)-HyperNEAT, Aprendizaje de caminatas.

\newpage

\section*{Abstract}

Gait generation for legged robots is an important task to allow an appropiate displacement in different scenarios, since hand-tuning design involves a big challenge due to the high computational efforts translated into processing power and time. This project study a popular neuroevolution method called HyperNEAT, which stands for Hypercube-based Neuroevolution of Augmenting Topologies. HyperNEAT is a generative encoding mechanism which evolves artificial neural networks using the principles of the algorithm of Neuroevolution of Augmenting Topologies (NEAT). It is a new technique for evolve large scale artificial neural networks using geometric regularities of the problem present in the network. Based on this method, this report proposes the implementation of the method called \(\tau\)-HyperNEAT, incorporating temporal concepts in a HyperNEAT neural network, introducing additional time delays in neural connections between neurons, allowing gait generation with behaviours and characteristics more similar to gaits found in nature. Subsequent to the implementation of both methods, these are tested in gaits generation of two robots with different number of degrees of freedom. The comparative analysis of the results reveals that, regarding quantitative variables of the experiment, there is no significant difference in performance obtained between HyperNEAT and \(\tau\)-HyperNEAT, but not from the qualitative point of view, obtaining notable differences in coordinating the movement of the limbs of each robot, producing more natural and complex gaits.

\section*{Keywords:} Artificial neural network, neuroevolution, HyperNEAT, \(\tau\)-HyperNEAT, Learning gaits.

\newpage

\section*{Glosario}

\begin{itemize}

\item[\textbf{ANN}] Artificial Neural Network (Red Neuronal Artificial)

\item[\textbf{IA}] Inteligencia Artificial

\item[\textbf{CPPN}] Compositional Pattern Producing Networks

\item[\textbf{NEAT}] NeuroEvolution of Augmenting Topologies

\item[\textbf{HyperNEAT}] Hypercube-based Neuroevolution of Augmenting Topologies

\end{itemize}
