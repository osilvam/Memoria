CPPNs es un método de codificación que permite describir directamente relaciones estructurales de una topología a través de la composición de funciones.

Desde una perspectiva estructura CPPNs y ANNs son muy similares debido a que métodos designados para evolucionar ANNs también pueden evolucionar CPPNs. En particular, el método de ``NeuroEvolution of Augmenting Topologies'' (NEAT) es una buena opción para evolucionar CPPNs porque NEAT incrementa la complejidad de las redes a medida que evolucionan generación tras generación, permitiendo que se generen cada vez regularidades mas elaboradas.

Un concepto base en la codificación en la naturaleza es que un pequeño numero de genes pueden codificar un gran numero de estructuras a través de la reutilización de genes.

La reutilización de genes se hace posible debido a que una estructura puede poseer un gran número de regularidades, tal como ocurre en la naturaleza. De no existir regularidades no se hace posible reproducir distintas partes de una estructura a partir de la misma información, perdiendo gran ventaja en la codificación. 

Identificar las caracteristicas generales de los patrones presentes en la naturaleza 

